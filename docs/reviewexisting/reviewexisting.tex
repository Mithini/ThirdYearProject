
A comprehensive review on existing book exchange platforms, book-availability and
metadata databases was performed. 20+ local e-tailers and 7 relevant international
e-tailers were reviewed. A list of book sites can be found in the Appendix (see
Section \ref{sec:appendixI}).

The study shows that while there are a plethora of existing book sales and exchange
platforms, both affiliated with major book retailers/e-tailers and independent, the
platforms themselves have a small selection of books, and are not updated frequently
enough to be useful to the end user. The stores also use rudimentary `newest first'
or `most popular' sorting algorithms, which are less effective in showing what the
user is most likely to be interested in.

International e-tailers such as Amazon and AbeBooks have a much larger selection of
books, both new and used, but do not provide competitive services to Sri Lankans due
to high shipping costs, long delivery times, and import restrictions.

Here's an overview of some of the globally existing websites:

\subsubsection{Amazon}
Amazon is a popular online Web-App that offers both new and used books for sale
through third-party sellers. Users can find a wide selection of used books in
various conditions. It offers a vast collection of books and a user-friendly
interface. This system has several subcategories that the books are classified
by department, format, author, promotions, prizes, languages, etc. The user
rating and book description part of amazon.com are one of the best features, as
it allows users to make an informed decision before purchasing a book. However,
the disadvantages of this website are, 1. E-commerce focus 2. Limited
transparency 3.lack of personalization 4. High transaction cost

\subsubsection{Goodreads}
Goodreads primarily serves as a social platform for readers, also some users
may offer books for trade or giveaway in discussion forums. This includes
details about authors, publication information, and user-generated content. The
site allows users to set and track reading goals for the year, which is a great
feature for people who like to challenge themselves to read more. It allows
users to create virtual bookshelves, rate books, and connect with other
readers. However, disadvantages of this are, 1. Limited transaction
capabilities 2. Incomplete listings 3. Complex selling process

\subsubsection{AbeBooks}
AbeBooks is a well-established online marketplace for books, including new,
used, rare, and out-of-print versions of books. Many independent sellers and
bookstores list their inventory on AbeBooks, which means you can find books
that may not be available on mainstream platforms. This website's interface is
user-friendly, and sellers on AbeBooks typically provide detailed descriptions
of the book's condition, edition, etc., as well as international shipping
options. This helps buyers make informed decisions. The website consists of
rare or antique books, often including first editions and signed copies.
Disadvantages of using this are, 1. Limited free listings 2. Commercial nature

\subsubsection{Bookberry.lk}
Bookberry is an online website that offers to sell used books online. Even
though not overly famous across Sri Lanka gives the facility to bank deposit or
cash on delivery option for buyers. Disadvantages are that it doesn't contain a
large collection of books and user interfaces are not friendly.

\subsubsection{UsedBooks.lk}
UsedBooks.lk is an online marketplace for books where you can find used and
rare books used by independent traders around Sri Lanka for sale. Website
contains detailed descriptions on the book's condition and photos uploaded by
the seller. They also have various categories of books but the drawback is that
since the books aren't categorized correctly the buyer finds trouble in sorting
through the catelog looking for the book he wants. Registration process is
fairly easy which creates oppotunities for fake sellers. Another notable
disadvantage is that even though books have been used the prices are overly
high which may discourage lot of buyers.

\subsubsection{bookswap.lk}
BookSwap.lk is free online platform for exchange used books or sell books free
online. You can buy second hand books, text books or used books by
institutions. Even though they have put up an extensive collection of books the
website is still completed as they haven't specified a payment method. Platform
encourages to meet the buyer personally rather than online transactions which
might not be appealing for some customers.